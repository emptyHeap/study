\documentclass[a4paper]{book}

\usepackage[T1,T2A]{fontenc}
\usepackage[utf8x]{inputenc}
\usepackage[russian]{babel}
\usepackage{listings}
\usepackage{longtable}
\usepackage{booktabs}
\usepackage{array}

\newcommand{\ttcode}[1]{%
    \texttt{#1}%
}
\newcommand{\ttcsname}[1]{%
    \ttcode{\char`\\#1}%
}

\begin{document}

\chapter{Overview of \LaTeX}

\begin{longtable}{p{7.5cm}|p{7.5cm}}
\hline
  Original & Перевод \\
\hline

\LaTeX\ is a system for typesetting documents. It was originally created by Leslie Lamport and is now maintained by a group of volunteers (http://latex-project.org). It is widely used, particulary for complex and technical documents, such as those involving mathematics.
&
\LaTeX\ это система для вёрстки документов. Изначально созданная Лесли Лампортом и сейчас поддерживающаяся группой волонтёров (http://latex-project.org). Она широко используется, удобна для сложных и технических документов, как те, которые включают математику.

\\

A \LaTeX\ user writes an input file containing text along with interspersed commands, for instance commands describing how the text should be formatted. It is implemented as a set of related commands that interface with Donald E. Knuth's \TeX\ typesetting program (the technical term is that \LaTeX\ is a \textit{macro package} for the \TeX\ engine). The user produces the output document by giving that input file to the \TeX\ engine.
&
Пользователи \LaTeX\ набирают файл, содержащий текст с вкраплениями комманд, например команд, описывающих то, как текст должен быть отформатирован. Это реализованно как набор связанных команд, которые взаимодействуют с программой для вёрстки текста \TeX\ Дональда Е. Кнута (выражаясь техническим языком \LaTeX\ это \textit{макропакет} для движка \TeX). Пользователь создаёт готовый документ, передавая этот входной файл движку \TeX.
\\
The term \LaTeX\ is also sometimes used to mean the language in which the document is marked up, that is, to mean the set of commands available to \LaTeX user.
&
Так же иногда термин \LaTeX\ используется для обозначения языка, на котором был размечен документ, то-есть для обозначения набора команд доступных пользователю \LaTeX.
\\
The name \LaTeX\ is short for "Lamport \TeX\ ". It is pronounced LAH-teck or LAY-teck, or sometimes LAY-tecks. Inside a document, produce the logo with \LaTeX. Where use of the logo is not sensible, such as in plain text, write it as 'LaTeX'.
&
Название \LaTeX\ это сокращение для "Лэмпорт \TeX". Оно произносится как ла-тэк или лэ-тэк, или, иногда, как лэ-текс. Воспроизвести логотип внутри документа можно с помощью \textbackslash LaTeX. Когда использование логотипа не имеет смысла, как например в неформатированном тексте, пишите его как 'LaTeX'.
\end{longtable}

\section{Starting and editing}

\begin{longtable}{p{7.5cm}p{7.5cm}}
\hline
Original & Перевод \\
\hline

\LaTeX\ files have a simple global structure, with a standard beginning and ending. Here is a "hello, world" example:
&
У файлов \LaTeX\ простая общая структура, с стандартным началом и концом. Вот пример "Привет мир"
\\

\begin{lstlisting}
\documentclass{article}
\begin{document}
Hello, \LaTeX\ world.
\end{document}
\end{lstlisting}

\\
Here, the 'article' is the so-called \textit{document class}, implemented in a file \verb|article.cls|. Any document class can be used. A few document classes are defined by \LaTeX\ itself, and vast array of others are widely available. See Chapter 3 [Document classes], page 8.
&
В этом примере 'article' это так называемый \textit{класс документа}, реализованный в файле \verb|article.cls|. Любой класс документа может быть использован. Несколько классов документов определены самим \LaTeX ом, и огромное множество других широко распространено. Смотри Главу 3 [Классы документов], страница 8.
\\
You can inclue other \LaTeX\ commands between the \verb|\documentclass| and  the \verb|\begin{document}| commands. This area is called the \textit{preamble}
&
Вы можете включать другие команды \LaTeX\ между командами \verb|\documentclass| и \verb|\begin{document}|. Эта область называется \textit{преамбулой}.
\\
The \verb|\begin{document} ... \end{document}| is so called \textit{environment}; the 'document' environment (and no others) is required in all \LaTeX\ documents (see Section 8.6 [document], page 41). \LaTeX\ provides many environments itself, and many more are defined separately. See Chapter 8 [Environments], page 37.
&
\verb|begin{document} ... \end{document}| это так называемое \textit{окружение}; Окружение 'документа' необходимо для всех \LaTeX\ документов (смотри Раздел 8.6 [документ], страница 41). \LaTeX\ сам предоставляет множество окружений, и намного больше определенно отдельно. Смотри главу 8 [Окружения], страница 37.
\\
The following sections discuss how to produce PDF or other output from a \LaTeX\ input file.
&
В следующем разделе обсуждается как получить PDF или другой вывод из входного \LaTeX\ файла
\\
\end{longtable}

\section{Output files}

\begin{longtable}{p{7.5cm}p{7.5cm}}
\hline
Original & Перевод \\
\hline

\LaTeX produces a main output file and at least two accessory files. The main output file's name ends in either .dvi or .pdf.
&
\LaTeX создаёт главный файл и по меньшей мере ещё два вспомогательных. Имя главного файла заканчивается либо на .dvi либо на .pdf.
\\
\textbf{.dvi} \newline
If \LaTeX\ is invoked with the system command \textbf{latex} then it produces a DeVice Independent file, with extension .dvi. You can view this file with a command such as \textbf{xdvi}, or convert it to a PostScript \textbf{.ps} file with \textbf{dvips} or to a Portable Document Format \textbf{.pdf} file with \textbf{dvipdfmx}. The contents of the file can be dumped in human-readable form with \textbf{dvitype}. A vast array of other DVI utility programs are available (http://mirror.ctan.org/dviware).
&
Если \LaTeX\ вызван с помощью системной команды \textbf{latex}, то он создаёт файл формата DeVice Independent, с расширением .dvi. Вы можете просмотреть этот файл с помощью команды вроде \textbf{xdvi} или преобразовать его в файл формата PostScript \textbf{.ps} с помощью \textbf{dvips} или в файл формата Portable Document Format \textbf{.pdf} с помощью \textbf{dvipdfmx}. Содержимое созданного файла может быть выгружено в человеко-читаемой форме с помощью \textbf{dvitype}. Доступно огромное множество других програм для работы с DVI (http://mirror.ctan.org/dviware).
\\
\textbf{.pdf} \newline
If \LaTeX\ is invoked via the system command pdflatex, among other commands (see Section 2.3 [\TeX\ engines], page 4), then the main output is a Portable Document Format (PDF) file. Typically this is a self-contained file, with all fonts and images included
&
Если \LaTeX\ вызван через системную команду pdflatex, наряду с другими (смотри раздел 2.3 [\TeX\ engines], страница 4), тогда главный получаемый файл будет иметь формат Portable Document Format (PDF). Это обычно самодостаточный файл, включающий все шрифты и изображения
\\
\LaTeX also produces at least two additional files
&
Так же \LaTeX создаёт по меньшей мере два дополнительных файла
\\
\textbf{.log} \newline
This transcript file contains summary information such as a list of loaded packages. It also includes diagnostic messages and perhaps additional information for any errors.
&
---НУЖОН ПЕРЕВОД СЛОВА TRANSCRIPT---. Также он включает отладочные сообщения и, возможно, дополнительную информацию об ошибках.
\\
\textbf{.aux} \newline
Auxiliary information is used by \LaTeX\ for things such as cross references. For example, the first time that \LaTeX\ finds a forward reference---a cross reference to something that has not yet appeared in the source---it will apear in the output as a doubled question mark ??. When the referred-to spot does not eventually appear in the source the \LaTeX\ writes its location information to this \textbf{.aux} file. On the next invocation, \LaTeX\ reads the location information from this file and uses it to resolve the reference, replacing the double question mark with the remembered location.
&
\textbf{.aux} \newline
Вспомогательная информация, используемая \LaTeX\ для таких вещей, как перекрёстные ссылки. Например, когда \LaTeX\ впервые находит прямую ссылку---перекрёстную ссылку на что-то что ещё не появилось в источнике---она отобразится в выводе как двойной знак вопроса ??. Если указанное место так и не появляется в источнике \LaTeX\ записывает информацию об его расположении в \textbf{.aux} файл. При последующем вызове, \LaTeX\ читает информацию о расположении из этого файла и использует её для определения ссылок, заменяя двойной знак вопроса на запомненное расположение.
\\
\LaTeX\ may produce yet more files, characterized by the filename ending. These include a \textbf{.lot} file that is used to make a list of figures, a \textbf{.lot} file used to make a list of tables, and a \textbf{.toc} file used to make a table of contents. A particular class may create others; the list is open-ended.
&
\LaTeX\ может создавать и другие файлы, отличающиеся расширением. К таким файлам относятся \textbf{.lot} файлы, которые используюся для создания списка фигур, \textbf{.lot} файл используемый для создания списка таблиц и \textbf{.toc} файл, используемый для создания оглавления. Особые классы могут создавать другие; приведённый список неокончательный.
\\
\end{longtable}

\section{\TeX\ engines}

\begin{longtable}{p{7.5cm}p{7.5cm}}
\hline
Original & Перевод \\
\hline
\LaTeX\ is defined to be a set of commands that are run by a \TeX\ implementation (see Chapter 2 [Overview], page3). This section gives a terse overview of the main programs.
&
\LaTeX\ определяется как набор команд, запускаемых реализациями \TeX. В этом разделе представлен краткий обзор основных из них.
\\
\textbf{pdflatex} \newline
In \TeX\ Live (http://tug.org/texlive), if \LaTeX\ is invoked via either the system command \textbf{latex} or \textbf{pdflatex}, then the pdf\TeX\ engine is run (http://ctan.org/pkg/pdftex). When invoked as \textbf{latex}, the main output is a \textbf{.dvi} fil; as \textbf{pdflatex}, the main output is a \textbf{.pdf} file. \newline
pdf\TeX\ incorporates the e-\TeX\ extensions to Knuth's original program (http://ctan.org/pkg/etex), including additional programming features and bi-directional typesetting, and has plenty of extensions of its own. e-\TeX\ is available on its own as the system command \textbf{etex}, but this is plain \TeX\ (and produces \textbf{.dvi}). \newline
In other \TeX\ distributions, \textbf{latex} may invoke e-\TeX\ rather than pdf\TeX.In any case, the e-\TeX\ extensions can be assumed to be available in \LaTeX.
&
В \TeX\ Live (http://tug.org/texlive), если \LaTeX\ запущен с помощью системной команды \textbf{latex} или \textbf{pdflatex}, то запускается движок pdf\TeX\ (http://ctan.org/pkg/pdftex). Если запущен \textbf{latex}, то главным выводом будет файл формата \textbf{.dvi}, если \textbf{pdflatex}, то формата .pdf. \newline
pdf\TeX\ включает расширение e-\TeX\ для оригинальной программы Кнута (http://ctan.org/pkg/etex), включая дополнительные возможности программирования и двухстороннюю вёрстку, а также множество собственных дополнений. e-\TeX\ доступен через свою системную команду \textbf{etex}, однако то простой \TeX\ (и создаёт \textbf{.dvi}). \newline
В других сборках \TeX\, \textbf{latex} может ссылаться на e-\TeX\ вместо pdf\TeX.В любом случае можно считать  расширения e-\TeX\ доступными в \LaTeX.
\textbf{pdflatex} \newline

\\
\textbf{lualatex} \newline
If \LaTeX\ is invoked via the system command \textbf{lualatex}, the Lua\TeX\ engine is run (http://ctan.org/pkg/luatex). This program allows code written in the scripting language Lua (http://luatex.org) to interact with \TeX's typesetting. Lua\TeX\ handles UTF-8 Unicode input natively, can handle Open-Type and TrueType fonts, and produces a \textbf{.pdf} file by default. The is also \textbf{dviluatex} to produce a \textbf{.dvi} file, but this is rarely used.
&
\textbf{lualatex} \newline
\\
\textbf{xelatex} \newline
If \LaTeX\ is invoked with the system command \textbf{xelatex}, the Xe\TeX\ engine is run (http://tug.org/xetex). Like Lua\TeX\, Xe\TeX\ natively supports UTF-8 Unicode and TrueType and OpenType fonts, though the implementation is completely different, mainly using external libraries instead of internal code Xe\TeX\ produces a \textbf{.pdf} file as output; it does not support DVI output. \newline
Internally, Xe\TeX\ creates an \textbf{.xdv} file, a variant of DVI, and translates that to PDF using the \textbf{(x)dvipdfmx} program, but this process is automatic. The \textbf{.xdv} file is only useful for debugging.
&
\textbf{xelatex} \newline
\\
Other variants of \LaTeX\ and \TeX\ exist, e.g., to provide additional support for Japanese and other languages
&

\end{longtable}

\section{\LaTeX\ command syntax}

\begin{longtable}{p{7.5cm}p{7.5cm}}
\hline
Original & Перевод \\
\hline

If the \LaTeX\ input file, a command name starts with a backslash character, \textbackslash. The name itself then consists of either (a) a string of letters or (b) a single non-letter. 
&
Если входной файл содержит команду, начинающуюся на символ обратной черты \textbackslash, то имя состоит либо (а) из строки состоящей из букв, либо (b) из одного не буквенного символа.
\\
\LaTeX\ commands names are case sensitive so that \verb|\pagebreak| differs from \verb|\Pagebreak| (the latter is not a standard command). Most commands are lowercase, but in any event you must enter all commands in the same case as the are defined.
&
Имена команд \LaTeX\ чувствительны к регистру, так \verb|\pagebreak| отличается от \verb|\Pagebreak| (последняя не является стандартной командой). Большинство команд это строчные буквы, но в любом случае вы должны вводить команды в том же регистве, как они были определенны.
\\
A command may be followed by zero, one, or more arguments. These arguments may be either required or optional. Required arguments are contained in curly braces, {...}. Optional arguments are contained in quare brackets, [...]. Generally, but not universally ifthe command accepts an optional argument, it comes first, before any required arguments.
&
За командой может следовать ни одного, один или более аргументов. Эти аргументы могут быть как необходимыми, так и не обязательными. Необходимые аргументы содержатся в фигурных скобках, {...}. Необязательные аргументы содержатся в квадратных скобках, [...]. В основном, но не всегда если команда принимает необязательные аргументы, то они идут первыми, перед всеми необходимыми аргументами.
\\
Inside of an optional argument, to use the character close square bracket (]) hide it inside curly braces, as in \verb|\item[closing bracket{]}]|. Similarly, if an optional argument comes last, with no required arguments after it, then to make the first character of the following text be an open square bracket, hide it inside curly braces.
&
Чтобы использователь закрывающуюся квадратную скобку (]) внутри необязательного аргумента спрячьте её между фигурных скобок, как в \verb|\item[closing bracket{]}]|. Похожим образом, если необязательный аргументы стоит последним, без необходимых аргументов после него для того чтобы сделать первый символ идущего следом текса открывающейся квадратной скобкой спрячьте её между фигурных скобок
\\
\LaTeX\ has the convention that some commands have a \textbf{*} form that is related to the form without \textbf{*}, such as \verb|\chapter| and \verb|\chapter*|. The exact difference in behaviour varies from command to command.
*
В \LaTeX\ есть соглашение, что у некоторых команд есть \textbf{*} форма, которая связана с формой без \textbf{*}, такие как \verb|\chapter| и \verb|\chapter*|. Разница их поведения отличается от команды к команде.
\\
This manual describes all accepted options and *-forms for the commands it covers (barring unintentional omissions, a.k.a. bugs).
&
Эта инструкция описывает все принимаемые параметры и *-формы для команд, которые она описывает (за исключением ненамеренных ошибок, так называемых багов).

\end{longtable}

\subsection{Environments}

\begin{longtable}{p{7.5cm}p{7.5cm}}
\hline
Original & Перевод \\
\hline
Symopsis: & \\
\begin{lstlisting}
	\begin{environment name}
		...
	\end{environment name}
\end{lstlisting}
\\
An area of \LaTeX\ source, inside of which there is a distinct behavior. For instance, for poetry in \LaTeX\ put the lines between \verb|\begin{verse}| and \verb|\end{verse}|.
&
Область кода \LaTeX\, со своим поведением врутри. Например, для того чтобы записать стих запишите строки между \verb|\begin{verse}| и \verb|\end{verse}|.
\\
\begin{lstlisting}
	\begin{verse}
		There once was a man from Nantucket \\
		...
	\end{verse}
\end{lstlisting}
\\
See Chapter 8 [Environments], page 37, for a list of environments.
&
\\
The \textit{environment name} at the begining must exactly match that at the end. This includes the case where \textit{environment name} ends in a star (*); bothe the \verb|\begin| and \verb|\end| texts must include the star.
&
\textit{Имя окрежения} (\textit{environment name}) в начале(begin) должено в точности совпадать с тем, тем что в конце(end). Это касается и случая, когда \textit{имя окружения} заканчивается символом астериск(*) - оба текста в \verb|\begin| и \verb|\end| должны включасть астериск.
\\
Environments may have arguments, including optional arguments. This example produces a table. The first argument is optional (and causes the table to be aligned on its top row) while the second argument is required (it specifies the formatting of columns).
&
Окружения могут иметь аргументы, включая необязательные аргументы. Этот пример создаёт таблицу. Первый аргумент опциональный (приводит к выравниванию таблицы по верхней строке), тогда как второй аргумент обязательный (необходим) (он опреденяет форматирование колонок).
\\
\begin{lstlisting}
	\begin{tabular}[t]{r|l}
		... rows of table ...
	\end{tabular}
\end{lstlisting}
\end{longtable}

\subsection{Commands declarations}

\begin{longtable}{p{7.5cm}p{7.5cm}}
\hline
Original & Перевод \\
\hline
A command that changes the value, or changes the meaning, of some other commands or parameter. For instance, the \verb|\mainmatter| command changes the setting of page numbers from roman numerals to arabic.
&
Команда, которая изменяет значение или изменяет смысл некоторой другой программы или параметра. Например, команда \verb|\mainmatter| изменяет параметры номера страницы с римских чисел на арабские.
\end{longtable}

\subsection{\ttcsname{makeatletter} and \ttcsname{makeatother}}

\begin{longtable}{p{7.5cm}p{7.5cm}}
\hline
Original & Перевод \\
\hline
Symopsis: & \\
\begin{lstlisting}
	\makeatletter
		... definition of commands with @ in their name ...
	\makeatletter
\end{lstlisting}
&
... определение комманд с @ в их имени ...
\\
Used to redefine internal \LaTeX\ commands. \verb|\makeatletter| makes the at-sign character @ have the category code of a letter, 11. \verb|\makeatother| sets the category code of @ to 12, its original value.
&
Используется для переопределения внутренних комманд \LaTeX\ . \verb|\makeatletter| меняет код кадегории символа @ на 11. \verb|\makeatother| устанавливает код категории @ на 12, его оригинальное значение.
\\
As each character is read by \TeX\ for \LaTeX, it is assigned a character category code, or \textit{catcode} for short. For instance, the backslash \verb|\| is assigned the catcode 0, for characters that start a command. These two commands alter the catcode assigned to @.
&
Каждому символу, прочитанному \TeX\ для \LaTeX\ он устанавливает код категории символа, или сокращённо \textit{catcode}. Например, обратной черте \verb|\| назначен catcode 0, для для символов начинающих команду. Эти две команды меняют catcode назначенный символу @.
\\ 
The alteration is needed because many of \LaTeX's commands use @ in their name, to prevent users from accidentally defining a command that replaces one of \LaTeX's own. Command names consist of a category 0 character, ordinarily backslash, followed by letters, category 11 characters (except that a command name can also consist of a category 0 character followed by a single non-letter symbol). So under the default category codes user-defined commands cannot contain an @. But \verb|\makeatletter| and \verb|\makeatother| allow users to define or redefine commands named with @.
&
Это изменение необходимо, так как множество комманд \LaTeX использует @ в их именах, чтобы предостеречь пользователей от случайного определения комманд, которые заменяют принадлежат \LaTeX. Имена команд состоят из символа категории 0, как правило обратной черты, за которой следуют буквы, символы с категорией 11 (за исключением команд чьи имена также могут содержать символ категории 0, за которым следует один символ не являющийся буквой). Так что в стандартной категории кодов пользовательские команды не могул содержать @, но \verb|\makeatletter| и \verb|\makeatother| разрешают пользователю определить или переопределить команды с символом @.
\\
Use these two commands inside a .tex file, in the preamble, when defining or redefining a command with @ in its name. Don't use them inside \textbf{.sty} or \textbf{.cls} files since the \verb|\usepackage| and \verb|\documentclass| commands set the at sign to have the character code of a letter.
&
Используйте две эти команды в .tex файле, в преамбуле, когда определяете или переопределяете команды с символом @ в их названии. Не используйте их внутри \textbf{.sty} и \textbf{.cls} файлах так как комманды \verb|\usepackage| и \verb|documentclass| устанавливаю код символа буквы для символа @.
\\
For a comprehensive list of macros with an at-sign in their names see \textbf{http://ctan.org/pkg/macros2e}. These macros are mainly intended to package or class authors
&
Для полного списка макросов включающих символ @ в их имена смотрите \textbf{http://ctan.org/pkg/macros2e}. Эти макросы в основном предназначены для авторов пакетов и классов.
\\
The example below is typical. In the user's class file is a command \verb|\thesis@universityname|. The user wants to change the definition. These three lines should go in the preamble, before the \verb|\begin{document}|.
&
Следующий пример типичен. В пользовательском файле классв есть команда \verb|thesis@universityname|. Пользователь хочет изменить определение. Следующие три строки должны быть добавлены в преамбулу, перед \verb|\begin{document}|.
\\
\begin{lstlisting}
	\makeatletter
	\revewcommand{\thesis@universityname}{Saint Michael's College}
	\makeatother
\end{lstlisting}
\\
\end{longtable}

\subsection{\ttcsname{\ @ifstar}}

\begin{longtable}{p{7.5cm}p{7.5cm}}
\hline
Original & Перевод \\
\hline
Synopsis:
\\
\begin{lstlisting}
	\newcommand{\mycmd}{\@ifstar{\mycmd@star}{\mycmd@nostar}}
	\newcommand{\mycmd@nostar}[non-starred command number of args]{
		body of non-starred command
	}
	\newcommand{\mycmd@star}[starred command number of args]{
		body of starred command
	}
\end{lstlisting}
\\
Many standard \LaTeX\ environments or commands have a variant with the same name but ending with a star character *, an asterisk. Examples are the table and table* environments and the \verb|\section| and \verb|\section*| commands.
&
Множество стандартных сред и комманд \LaTeX\ обладают вариантами с одинаковым именем, заканчивающихся на звезду *, астериск. Примерами являются среды table и table* и команды \verb|\section| и \verb|\section*|.
\\
When defining environments, following this pattern is straightforward because \verb|\newenvironment| and \verb|\renewenvironment| allow the environment name to contain a star. For commands the situation is more complex. As in the synopsis above, there will be a user-called command, given above as \verb|\mycmd|, which peeks ahead to see if it is followed by a star. For instance, \LaTeX\ does not really have a \verb|\section*| command; instead, the \verb|\section| command peeks ahead. This command does not accept arguments but instead expands to one of two commands that do accept arguments. In the synopsis these two are \verb|\mycmd@nostar| and \verb|\mycmd@star|. They could take the same number of arguments or a different number or no arguments at all. As always in a \LaTeX\ document a command using at-sign @ must be enclosed inside a \verb|\makeatletter ... \makeatother| block.
&
При определении сред, следование этим шаблонам не вызывает затруднений, потому что \verb|newenvironment| и \verb|\renewenvironment| позволяют названию среды содержать астериск. Для команд ситуация более сложная. Как в конспекте(примере) выше, когда будет вызвана пользовательская команда, названная выше \verb|\mycmd|, которая взглянет вперёд чтобы узнать следует ли за ней звезда. Например, в \LaTeX\ на самом деле нет команды \verb|\section*|, вместо неё команда \verb|\section| взглянит вперёд. Эта команда не принимает аргументы, но вместо этого разворачивается до одной из двух команд, которые принимают. В кратком обзоре это команды \verb|\mycmd@nostar| и \verb|mycmd@star|. Они могут принимать одинаковое количество аргументов или разное, или вообще не принимать аргументов. Как обычно в документе \LaTeX\ команды использующие @ должны быть заключены в блок \verb|\makeatletter ... \makeatother|.
\\
This example of \verb|\@ifstar| defines the command \verb|\ciel| and a variant \verb|\ciel*|. Both have one required argument. A call to \verb|\ciel{night}| will return "starry night sky" while \verb|\ciel*{blue}| will return "starry not blue sky".
&
Этот пример использования \verb|\@ifstar| определяет команду \verb|\ciel| и модификацию \verb|\ciel*|. Они обе имеют один необходимый аргумент. Вызов \verb|\ciel{night}| вернёт "starry night sky" тогда как \verb|\ciel*{blue}| вернёт "night not blue sky".
\\
\begin{lstlisting}
	\newcommand*{\ciel@unstarred}[1]{starry #1 sky}
	\newcommand*{\ciel@starred}[1]{starry not #1 sky}
	\newcommand*{\ciel}{\@ifstar{\ciel@starred}{\ciel@unstarred}}
\end{lstlisting}
\\
In the next example the starred variant takes a different number of arguments than does the unstarred one. With this definition, Agent 007's \verb|"My name is \agentsecret*{Bond},| \verb|\agentsecret{James}{Bond}"| is equivalent to \verb|"My name is \textsc{Bond}, \textit{James}| \verb| textsc{Bond}."|.
&
В следующем примере модификация со звездой принимает отличное от команды без звезды количество аргументов. Со следующим определением, Агента 007 результат \verb|"My name is \agentsecret*{Bond},| \verb|\agentsecret{James}{Bond}"| соответствует \verb|"My name is \textsc{Bond}, \textit{James}| \verb| textsc{Bond}."|.
\\
\begin{lstlisting}
	\newcommand*{\agentsecret@unstarred}[2]{\textit{#1} \textsc{#2}}
	\newcommand*{\agentsecret@starred}[1]{\textsc{#1}}
	\newcommand*(\agentsecret}{\@ifstar{\agentsecret@starred}
		{\agentsecret@unstarred}}
\end{lstlisting}
\\
There are two sometimes more convinient ways to accomplish the work of \verb|\@ifstar|. The suffix package allows the construct \verb|\newcommand\mycommand{unstarred version}| followed by \verb|\WithSuffix\newcommand\mycommand| \verb|*{starred version}|. And \LaTeX3 has the xparse package that allows this code.
&
Есть два порой более удобных способа совершить работу \verb|@ifstar|. Суффиксный пакет позволяет построить составить \verb|newcommand\mycommand{unstarred version}| после \verb|\WithSuffix\newcommand\mycommand| |verb|*{starred version}|. А также \LaTeX3 включает пакет xparse, который позволяет использовать следующий код.
\\
\begin{lstlisting}
	\NewDocumentCommand\foo{s}{\IfBooleanTF#1
		{starred version}%
		{unstarred version}%
	}
\end{lstlisting}
\\
\end{longtable}

\subsection{}

\begin{longtable}{p{7.5cm}p{7.5cm}}
\hline
Original & Перевод \\
\hline
\end{longtable}

\end{document}
